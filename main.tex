\documentclass{article}
\usepackage{pgfplots}
\usepackage[german]{babel}

\usepackage{geometry,lipsum}% http://ctan.org/pkg/{geometry,lipsum}

% default
\geometry{margin=1in}% 1in margin
%\geometry{margin=1cm}% 1cm margin




\begin{document}

\begin{center}
    \huge \textbf{Rutschenmodellierer}\\
    \footnotesize{Aufgabe erstellt von Jonas, Paul, Dominik}
\end{center}

\vspace{1.5cm}

\textbf{Aufgabenstellung:}

\vspace{0.2cm}

Die beliebteste Attraktion der Stadt Bielefeld im Sommer, wenn die Temperatur über 30°C liegt, ist die Rutsche im Südschwimmbad.

Der Bademeister Jochen Jürgen hat sich aus Langeweile wegen der wenigen Kundschaft in den Morgenstunden eine Funktion $r(x)$ überlegt, die im ersten Quadranten eines kartesischen Koordinatensystems den Längenquerschnitt der Rutsche modelliert. 

Von der Baufirma des Auftraggebenden Schwimmbecken Liebreich GmbH hat er außerdem die Modellierung dieses Beckens mit der Funktion $b(x)$ erhalten:\\

\vspace{0.2cm}

$r(x) = -0,1x^4 - 0,5x^3 - 0,1x^2$ \\

$b(x) = 0,2x^2 - 5$\\

In fleißiger Arbeit hat sich der 20-jährige Schwimmwart die Funktion $b(x)$ dabei so modelliert, dass sie den Boden hinter der Rutsche im vierten Quadranten (ebenfalls im Querschnitt) anzeigt.

\vspace{2cm}

bsp plot:\\
\begin{figure}[h]
    \centering
\begin{tikzpicture}
\begin{axis}[
    axis lines = middle,
    xlabel = $x$,
    ylabel = $y$,
    xmin=-50, xmax=50,
    ymin=-50, ymax=50,
]
\addplot [
    domain=-50:50,
    samples=100,
    color=blue,
    ]
    {1/3*x^2};
\end{axis}
\end{tikzpicture}

\caption{My plot}
    \label{fig:myplot}
\end{figure}

a) [...]
\vspace{1cm}


Die Rutsche im Südschwimmbad wurde von der Rutschenfirma bla entwickelt und hergestellt. Der Rutschentyp, für den sich das Südschwimmbad entschieden hat, fällt in eine Reihe von Rutschen, bei denen der Rutschvorgang vereinfacht mit der Kurvenschar 
$$f_{a}(x) = 2a \cdot ln(x) - 0,2x, a > 0, x 	\in \mathbb{R}$$
modelliert wird. Auf der x1-Achse wird die Zeit in Sekunden modelliert; die x2-Achse zeigt die zurückgelegten Meter an. Nachdem ein Kunde Bedingungen an seine Rutsche gestellt hat, z.B. über Länge und Geschwindigkeit, kann die Firma einen genauen Parameter a der Schar festlegen, welcher diese entsprend respektiert. Es wird angenommen, dass ein solcher Parameter immer für die gegebene Schar gefunden werden kann. Außerdem werden Reibung und Gewicht des Rutschenden vernachlässigt.\\
\vspace{1cm}\\
b) \\
\vspace{0.05cm}\\
\textit{Bestimmen Sie rechnerisch die allgemeine Nullstelle und den allgemeinen Hochpunkt der Schar.}\\ 
(ggf.: Hinweis: Auf den Nachweis der hinreichden Bedingung kann verzichtet werden)\\
\vspace{0.05cm}\\
\textit{Begründen Sie nur anhand von $f^{\prime}_{a}(x)$ den Zeitpunkt, zu dem die Rutschgeschwindkeit des Rutschenden am größten ist.}\\
\vspace{0.05cm}\\
\textit{Beschreiben Sie die Bedeutung des Hochpunktes im Sachzusammenhang.}\\
\vspace{0.05cm}\\
\textit{Bestimmen Sie die durchschnittliche Geschwindigkeit des Rutschenden im Intervall (Nst., HP).}

{\raggedleft(6 + 2 + 2 + 4 Punkte)\\\vspace{1cm}\par}

Für die Moddelierung des Rutschvorgangs wird nun das Intervall (Nst., HP) betrachtet.
\vspace{1cm}\\
c)\\
\vspace{0.05cm}\\
\textit{Bestimmen Sie für a = 5 die momentane Geschwindigkeit des Rutschenden drei Sekunden nach Rutschbeginn.} (Hinweis: Von Reibung und Gewicht des Rutschenden kann abgesehen werden.)



\vspace{4cm}
- 

- max steigung tempo zu hoch für tester olaf

-durchschnittl geschw, meter




\end{document}