\documentclass{article}
\usepackage{pgfplots}
\usepackage[german]{babel}

\usepackage{geometry,lipsum}% http://ctan.org/pkg/{geometry,lipsum}

% default
\geometry{margin=1in}% 1in margin
%\geometry{margin=1cm}% 1cm margin




\begin{document}

\begin{center}
    \huge \textbf{Rutschenmodellierer}\\
    \footnotesize{Aufgabe erstellt von Jonas, Paul, Dominik}
\end{center}

\vspace{1.5cm}

\textbf{Aufgabenstellung:}

\vspace{0.2cm}

Die beliebteste Attraktion der Stadt Bielefeld im Sommer, wenn die Temperatur über 30°C liegt, ist die Rutsche im Südschwimmbad.

Der Bademeister Jochen Jürgen hat sich aus Langeweile wegen der wenigen Kundschaft in den Morgenstunden eine Funktion $r(x)$ überlegt, die im ersten Quadranten eines kartesischen Koordinatensystems den Längenquerschnitt der Rutsche modelliert. 

Von der Baufirma des Auftraggebenden Schwimmbecken Liebreich GmbH hat er außerdem die Modellierung dieses Beckens mit der Funktion $b(x)$ erhalten:\\

\vspace{0.2cm}

$r(x) = -0,1x^4 - 0,5x^3 - 0,1x^2$ \\

$b(x) = 0,2x^2 - 5$\\

In fleißiger Arbeit hat sich der 20-jährige Schwimmwart die Funktion $b(x)$ dabei so modelliert, dass sie den Boden hinter der Rutsche im vierten Quadranten (ebenfalls im Querschnitt) anzeigt.

\vspace{2cm}

bsp plot:\\
\begin{figure}[h]
    \centering
\begin{tikzpicture}
\begin{axis}[
    axis lines = middle,
    xlabel = $x$,
    ylabel = $y$,
    xmin=-50, xmax=50,
    ymin=-50, ymax=50,
]
\addplot [
    domain=-50:50,
    samples=100,
    color=blue,
    ]
    {1/3*x^2};
\end{axis}
\end{tikzpicture}

\caption{My plot}
    \label{fig:myplot}
\end{figure}

a) [...]
\vspace{1cm}

b)\\

Die Firma, die die Rutschen herstellt, modelliert den Rutschvorgang mit der Kurvenschar

\vspace{0.2cm}

\\ $f(x) = 2a * ln(x) - 0,2x, a < 0, x 	\in \mathbb{R}$ \\.




\end{document}